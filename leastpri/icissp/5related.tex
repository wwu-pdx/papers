Given the importance of securing the cloud, there have been a variety of cloud security exercises that have been built.
Offensive security exercises such as Flaws and Flaws2~\cite{flaws,flaws2} are CTF exercises that are hosted on vulnerable
Amazon Web Services (AWS) projects.  Each one takes students through a sequence of exploits in which they use initial access to the project
via a vulnerable web application or an open storage bucket to escalate their privileges throughout the project.  Both CTFs
utilize a hint system to ensure players can make progress quickly.  While players must have an AWS account to
play, most of the resources for the CTF are hosted by the CTF developers.   Similarly, CloudGoat \cite{cloudgoat} and OWASP ServerlessGoat \cite{serverlessgoat}
are exercises that have students exploit sequences of common cloud vulnerabilities such as authentication bypass and misconfigured IAM settings.
While Flaws2 contains a defender path in which players perform a post-mortem on the offensive path, the CTF exercises are mainly focused
on offensive security where the player acts as an attacker.

On Google Cloud Platform (GCP), Thunder CTF \cite{thunder-ctf} covers an extensive range of GCP based vulnerabilities.  As with the prior AWS exercises, it too
is focused on offensive security where students practice attacking vulnerable projects.  While not structured as a CTF, sequences of QWIKLABS~\cite{QWIKLABS} 
that are focused on GCP do focus on having users practice securing their cloud projects.  In these labs, students practice creating service accounts, attaching
roles and granting permissions to them.  The labs consist mostly of guided walkthroughs where a secure configuration is organically generated and complements
the Least Privilege exercises in which a pre-existing, vulnerable configuration is being fixed.

%% Existing services are available to help achieve PoLP with less effort by monitoring logs and configuration. For example, Google IAM Recommender \cite{GoogleLstRec} provides safe, in-context, and actionable changes to IAM policies that move a project towards PoLP and don’t require lots of manual effort. However, it will have to gather enough data to be able to make suggestions. This limitation makes Google IAM Recommender as well as a lot of other third party tools impractical for education purposes.