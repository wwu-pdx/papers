\noindent Developers and researchers have actively worked on building cloud security CTF exercises or labs. Existing AWS CTF platforms are consist of  Flaws \cite{flaws}, Flaws2 \cite{flaws2}, CloudGoat \cite{cloudgoat} and OWASP ServerlessGoat \cite{serverlessgoat}. These platforms demonstrate common security issues caused by various reasons throughout development, including authentication bypass and over-privileged access(misconfigured IAM). Most CTF exercises are offensive style, meaning the player will act as an attacker. It is the players’ mission to identify vulnerabilities, and exploit the way to the ‘secret’. Flaws2 contains both offensive and defensive style of CTF. Thunder CTF \cite{thunder-ctf} covers an extensive range of GCP based labs, however best practice is not designed in any level. QWIKLABS \cite{QWIKLABS} seems to be the only platform that builds security labs regarding both AWS and GCP. Their courses and labs aim to help users gain cloud knowledge, practice building projects and get familiar with cloud services as quickly as possible. Although some QWIKLABS projects do require creating service account, attaching roles and granting permissions, they issue temporary cloud credentials for players, so security has never been their main concern.

Existing services are available to help achieve PoLP with less effort by monitoring logs and configuration. For example, Google IAM Recommender \cite{GoogleLstRec} provides safe, in-context, and actionable changes to IAM policies that move a project towards PoLP and don’t require lots of manual effort. However, it will have to gather enough data to be able to make suggestions. This limitation makes Google IAM Recommender as well as a lot of other third party tools impractical for education purposes.