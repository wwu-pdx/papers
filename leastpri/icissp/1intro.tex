\noindent Due to the agility, scalability, and potential cost savings that
cloud computing offers, businesses are in the process of shifting their workloads
to public cloud providers. According to a recent report~\cite{Flexera2020}, enterprises are running close to half of of their workloads and storing close to half of their data in public clouds with most planning to increase their usage of cloud infrastructure in the future.
% Clouds allow organizations to reduce startup costs and maintenance tasks in infrastructure (hardware and network) layers through virtualization. However, integration, both between cloud services and legacy systems and between various services from different cloud resources and platforms, is quite complex \cite{Baron2019}. 
% The findings in the same Flexera report \cite{Flexera2020} indicate that 93\% of enterprises are embracing multi-cloud solutions (involving multiple public and/or private clouds), among which 87\% are taking a hybrid approach. Services from different platforms that are not always compatible with each other can pose data interoperability and ownership issues. The rapid adoption of newer technologies such as containers and serverless frameworks also contributes to cloud complexity\cite{Sharrm}. As a result, other than cloud computing itself, IAM becomes more abstruse in the cloud compared to it in legacy IT environments. The requirements to understand and configure effective access for different applications, services and platforms exacerbate the problem. 
Unfortunately, with cloud computing comes a level of complexity that is far beyond what
typical organizations are accustomed to handling with their current IT environments.
Complicated fine-grained permissions and role-based access control applied to both legacy 
virtual machines and networks as well as to modern container-based and serverless platforms has 
made the practice of security in the cloud more abstruse~\cite{Sharrm}, raising significant
security and privacy concerns~\cite{Takabi2010}.

As a result, studies~\cite{Ermetic2020,IBMSecurity2020} have revealed that nearly 80\% of companies have experienced at least one cloud data breach and that one of the top threats
comes from improper Identity and Access Management (IAM) configurations (61\%).
Cases of actual breaches stemming from misconfigured or over-provisioned IAM
policies are common.
In the Capital One breach in July 2019, an attacker gained access to an EC2 instance with
overprovisioned IAM privileges in order to access sensitive information stored in a multitude
of AWS S3 storage buckets~\cite{Parimi2019}. 
In the AutoClerk breach, an Elastic search database hosted in AWS was left open, leaking
the personal details of hotel guests, US government and military personnel~\cite{Fawkes2020}.
Finally, in August 2020, two insiders at Twitter abused their overprovisioned privileges to
user data to collect and exfiltrate the personal information of high value users for the 
government of Saudi Arabia \cite{Newman2019}.

It is essential in any environment to always practice the principle of least privilege (PoLP)~\cite{Saltzer1974}. The concept of PoLP developed on the Multics system is that users or processes should only have the minimum privileges which are needed to perform its intended 
work.  In cloud deployments, properly minimizing access to resources is considered
a best practice that is essential for mitigating security risks~\cite{Sanders2018}. 
Unfortunately, it is a practice that many companies find difficult to prioritize.

% MOVE TO SECTION 2
%IAM can provide an adequate level of protection for all resources and data through rules and policies which are enforced on users via various techniques such as enforcing login password, assigning privileges to the users and provisioning user accounts \cite{AlmullaSameeraAbdulrahmanandYeun2010}.
%IAM ensures authentication, which validating the identity of users or systems. For example, authentication between services involves in verifying the access request to the information which served by another service \cite{AlmullaSameeraAbdulrahmanandYeun2010}.
%After authentication succeeds, authorization process will determine the privileges grant to legitimate users and enforce the security policies.

Specifically, cloud systems are now so complex that developers are reluctant to modify security
settings once ``it works''. As a consequence, any security errors that exist at deployment are
rarely fixed.  A report on DevOps security has found that only 4\% of security issues found in 
production are dealt with after development~\cite{Foremski}. Compounding this problem is the
fact that developers often have limited experience with PoLP and are often introduced to cloud 
services using guides and walkthroughs that de-emphasize good security practices in favor of 
fast and  frictionless adoption of the service.  In one such case, documentation suggests that 
the developer assign the ``Owner'' role to an account which gives the maximum permissions
possible, when in fact, recent changes to the platform have made the need for the role
unnecessary\cite{GoogleVis}.  If adveraries are then able to compromise the service,
they are given complete access to all of project's cloud resources.

%  is unnecessary for the 
% An interesting case study is that cloud service provider sometimes suggests using Owner role that granted with maximum permission in their documentation (Quickstart: Setup the Vision API) . Based on the developer behavior we know, this could generate a potential attacking surface in the future.
% Developers need guidance or training for secure coding, however, nearly 70\% of developers expressed that they get little help in GitLab's 2019 Global Developer Report  \cite{Gitlab2019}.

While applying the principle of least privilege is critical to cloud security, there are
few exercises that allow developers to practice it.  Many cloud-based ``capture-the-flag''
exercises that are available online allow one to instead practice identifying
vulnerabilities in cloud deployments and then exploiting them to achieve a goal. 
Such offensive-minded CTFs such as Flaws~\cite{flaws,flaws2}, Cloud Goat~\cite{cloudgoat},
and Serverless Goat~\cite{serverlessgoat} are available for Amazon Web Services while Thunder CTF~\cite{thunder-ctf} provides similar functionality for Google Cloud Platform.
Unfortunately, while the exercises show us what happens when permissions are over-provisioned,
they do not afford opportunites for us to practice reducing them. To address this, this
paper describes the design, implementation, and evaluation of a Least Privilege CTF, a set of CTF exercises for helping one practice preventing security issues in cloud deployments
via levels in which players apply least privileges onto deployments running on Google Cloud 
Platform (GCP).
 
% based exercises \cite{thunder-ctf} \cite{QWIKLABS}.  However, we find that none of them is destined for the purpose of   training developers on how to implement PoLP in cloud environment. 

Section 2 provides a tutorial on the basics of Identity and Access Management control on
GCP. Given this, Section 3 describes the comprehensive design of the Least Privilege CTF and its levels. Section 4 describes the evaluation of an initial deployment in an advanced elective course in our program.  Finally, Section 5 presents related work and Section 6 concludes.