\documentclass[a4paper,twoside]{article}

\usepackage{epsfig}
\usepackage{subcaption}
\usepackage{calc}
\usepackage{amssymb}
\usepackage{amstext}
\usepackage{amsmath}
\usepackage{amsthm}
\usepackage{multicol}
\usepackage{pslatex}
\usepackage{apalike}
\usepackage{SCITEPRESS}     % Please add other packages that you may need BEFORE the SCITEPRESS.sty package.


\begin{document}


\title{Game to Dethrone: A Least Privilege CTF}

%%\author{\authorname{Wenjing Wu\sup{1}\orcidAuthor{0000-0000-0000-0000}, %%Second Author Name\sup{1}\orcidAuthor{0000-0000-0000-0000} and Third Author %%Name\sup{2}\orcidAuthor{0000-0000-0000-0000}}
%%\affiliation{\sup{1}Institute of Problem Solving, XYZ University, My Street, %%MyTown, MyCountry}
%%\affiliation{\sup{2}Department of Computing, Main University, MySecondTown, %%MyCountry}
%%\email{\{f\_author, s\_author\}@ips.xyz.edu, t\_author@dc.mu.edu}
%%}

\author{\authorname{Wenjing Wu, Wu-chang Feng}
\affiliation{Department of Computer Science, Portland State University}
}

\keywords{cloud security, CTF, least privilege, IAM}
\abstract
{As more businesses integrating into the cloud environment, the importance of following the principle of least privilege (PoLP) to mitigate security risks significantly increases. The fact that both identity and access management (IAM) and infrastructure itself in cloud environments are sophisticated, and the lack of CTF (Capture-the-Flag) exercises pivot around IAM best practices make the task of understanding the concepts and following PoLP onerous. This paper describes Least Privilege CTF, a series of Google Cloud based labs that can be quickly deployed at minimal cost, to assist comprehension and illustrate the process of implementing PoLP.
}

\onecolumn \maketitle \normalsize \setcounter{footnote}{0} \vfill

\section{\uppercase{Introduction}}
\label{sec:introduction}
\input 1intro.tex


\section{\uppercase{Google Cloud IAM}}
\label{sec:gcpiam}
\input 2iam.tex

\section{Least Privilege CTF}
\label{sec:ctf}
\input 3ctf.tex

\section{Evaluation}
\label{sec:eval}
\input 4eval.tex

\section{Related Work}
\label{sec:related}
\input 5related.tex

\section{Conclusion}
\label{sec:concl}
\input 6concl.tex

\section{Acknowledgments}
\noindent This material is supported by the National Science Foundation under Grant No. 1821841. Any
opinions, findings, and conclusions or recommendations expressed in this material are those of the author and do not necessarily reflect the views of the National Science Foundation.

\bibliographystyle{apalike}
{\small
\bibliography{ref-base}}

\end{document}

