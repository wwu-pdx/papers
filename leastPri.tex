%%
%% This is file `sample-lualatex.tex',
%% generated with the docstrip utility.
%%
%% The original source files were:
%%
%% samples.dtx  (with options: `sigconf')
%% 
%% IMPORTANT NOTICE:
%% 
%% For the copyright see the source file.
%% 
%% Any modified versions of this file must be renamed
%% with new filenames distinct from sample-lualatex.tex.
%% 
%% For distribution of the original source see the terms
%% for copying and modification in the file samples.dtx.
%% 
%% This generated file may be distributed as long as the
%% original source files, as listed above, are part of the
%% same distribution. (The sources need not necessarily be
%% in the same archive or directory.)
%%
%% The first command in your LaTeX source must be the \documentclass command.
\documentclass[sigconf]{acmart}

%%%% As of March 2017, [siggraph] is no longer used. Please use sigconf (above) for SIGGRAPH conferences.

%%%% As of May 2020, [sigchi] and [sigchi-a] are no longer used. Please use sigconf (above) for SIGCHI conferences.

%%%% Proceedings format for SIGPLAN conferences 
% \documentclass[sigplan, anonymous, review]{acmart}

%%%% Proceedings format for conferences using one-column small layout
% \documentclass[acmsmall,review]{acmart}

%%
%% \BibTeX command to typeset BibTeX logo in the docs
\AtBeginDocument{%
  \providecommand\BibTeX{{%
    \normalfont B\kern-0.5em{\scshape i\kern-0.25em b}\kern-0.8em\TeX}}}



%%
%% Submission ID.
%% Use this when submitting an article to a sponsored event. You'll
%% receive a unique submission ID from the organizers
%% of the event, and this ID should be used as the parameter to this command.
%%\acmSubmissionID{123-A56-BU3}

%%
%% The majority of ACM publications use numbered citations and
%% references.  The command \citestyle{authoryear} switches to the
%% "author year" style.
%%
%% If you are preparing content for an event
%% sponsored by ACM SIGGRAPH, you must use the "author year" style of
%% citations and references.
%% Uncommenting
%% the next command will enable that style.
%%\citestyle{acmauthoryear}

%%
%% end of the preamble, start of the body of the document source.
\begin{document}

%%
%% The "title" command has an optional parameter,
%% allowing the author to define a "short title" to be used in page headers.
\title{Least Privilege CTF}

%%
%% The "author" command and its associated commands are used to define
%% the authors and their affiliations.
%% Of note is the shared affiliation of the first two authors, and the
%% "authornote" and "authornotemark" commands
%% used to denote shared contribution to the research.
\author{Wenjing Wu}
\email{trovato@corporation.com}
\email{wwu@pdx.edu}
\affiliation{%
  \institution{Portland State University}
  \streetaddress{1900 SW 4th Ave}
  \city{Portland}
  \state{Oregon}
  \postcode{97201}
}



%%
%% By default, the full list of authors will be used in the page
%% headers. Often, this list is too long, and will overlap
%% other information printed in the page headers. This command allows
%% the author to define a more concise list
%% of authors' names for this purpose.
\renewcommand{\shortauthors}{Trovato and Tobin, et al.}

%%
%% The abstract is a short summary of the work to be presented in the
%% article.
\begin{abstract}
 To do ...
\end{abstract}



%%
%% Keywords. The author(s) should pick words that accurately describe
%% the work being presented. Separate the keywords with commas.
\keywords{cloud security, CTF, least privilege, IAM}


%%
%% This command processes the author and affiliation and title
%% information and builds the first part of the formatted document.
\maketitle

\section{Introduction}
%% subsections are used to keep track of outlines and will be removed after..
\subsection{cloud and its complexity}
Advantages bestowed by cloud technologies, such as efficiency, agility, scalability and cost savings, are crucial for business success in today’s market. According to the 2020 Flexera State of the Cloud Report \cite{Flexera2020}, enterprises are running 48\% of their workloads and storing 45\% of their data in public clouds and planning to increase workload and data-related cloud utilization by 9\% and 8\% respectively over the next 12 months. This trend will continue to grow.

As defined by The US National Institute of Standards and Technology \cite{NIST2016},
Cloud computing is a model for enabling ubiquitous, convenient, on-demand network access to a shared pool of configurable computing resources (e.g., networks, servers, storage, applications, and services) that can be rapidly provisioned and released with minimal management effort or service provider interaction.

Clouds allow organizations to reduce startup costs and maintenance tasks in infrastructure (hardware and network) layers through virtualization. However, integration, both between cloud services and legacy systems and between various services from different cloud resources and platforms, is quite complex \cite{Baron2019}. 
The findings in the same Flexera report \cite{Flexera2020} indicate that 93\% of enterprises are embracing multi-cloud solutions (involving multiple public and/or private clouds), among which 87\% are taking a hybrid approach. Services from different platforms that are not always compatible with each other can pose data interoperability and ownership issues.
The rapid adoption of technologies such as containers and serverless frameworks also contributes to cloud complexity. In the State of DevSecOps report \cite{Sharrm} , containers are being used by 84\% of the organizations surveyed and 41\% are using serverless frameworks.

Identity and access control management(IAM) becomes a more complex problem in the cloud as a result of resource extensibility and diversity compared to legacy IT environments. The requirements to understand and configure effective access for different applications, services and platforms exacerbate the problem.



\subsection{Cloud Security}
Complexities mentioned above, along with the Unique architectural features, raise various security and privacy concerns \cite{Takabi2010}. A Study \cite{Ermetic2020} commissioned by Ermetic, IDC revealed that nearly 80\% of companies experienced at least one cloud data breach during 2019 and the first half of 2020 .and that the top three cloud security threats are misconfiguration of production environments (67 \%), lack of visibility into access in production environments (64 \%) and improper IAM and permission configurations (61\%). According to IBM's Cost of Data Breach Report 2020 \cite{IBMSecurity2020}, misconfigurations were exploited in 19\% of malicious breaches with a cost of \$4.41 million . This number is 14\% higher than the average cost of data breach.
\subsection{Cloud IAM}
\subsection{Least Privilage}
\subsection{why Least Privilage CTF}

\subsection{Section description}
Section description to do ...
\section{Google Cloud IAM}

\section{Least Privilege CTF}

\subsection{Design Goal}
\subsection{Implementation}
\subsubsection{Thunder CTF}
\subsubsection{Cloud function based and UI}
\subsubsection{Levels and examples}
\section{Evaluation}
\section{Conclusion}
\subsection{Related Work}
Google cloud IAM Recommender ....


\section{Acknowledgments}

This material is supported by the National Science Foundation under Grant No. 1821841. Any
opinions, findings, and conclusions or recommendations expressed in this material are those of the
author and do not necessarily reflect the views
of the National Science Foundation.




%%
%% The next two lines define the bibliography style to be used, and
%% the bibliography file.
\bibliographystyle{ACM-Reference-Format}
\bibliography{ref-base}

%%
%% If your work has an appendix, this is the place to put it.
\appendix


\end{document}
\endinput
%%
%% End of file `sample-lualatex.tex'.
